\documentclass[12pt]{article} 
\usepackage{ucs} 
\usepackage[utf8x]{inputenc}
\usepackage[russian]{babel} 
\usepackage[left=2cm,right=2cm,top=0,5cm,bottom=0,5cm]{geometry}
\date{}
\title{\bf 17 уравнений, изменивших ход истории} 
\begin{document} 

\maketitle
\begin{enumerate}
 \item Теорема Пифагора \hfill \begin{minipage}[t]{100mm} $a^{2}+b^{2} = c^{2}$ \hfill Пифагор 530 до нашей эры \end{minipage}
 \item Логарифмы  \hfill \begin{minipage}[t]{100mm} $\log xy = \log x+\log y$ \hfill Джон Напиер, 1610 \end{minipage} 
 \item Формула Ньютона-Лейбенца \hfill \begin{minipage}[t]{100mm} $\frac{df}{dt} = \lim_{h \to 0}\frac {f(t+h)-f(t)}{h} $ \hfill Исаак Ньютон, 1668 \end{minipage}
 \item   Закон гравитации \hfill \begin{minipage}[t]{100mm} $F = G \frac{m_{1}m_{2}}{r^{2}}$ \hfill Исаак Ньютон, 1687 \end{minipage}
 \item \noindent
  \begin{minipage}[t]{45mm} Мнимая единичка или корень из -1 \end{minipage} \hfill \begin{minipage}[t]{100mm}  $\imath^{2} = -1$ \hfill Леонард Эйлер, 1750 \end{minipage}
 \item \noindent
  \begin{minipage}[t]{45mm} Эйлерова \\характеристика\\ \end{minipage} \hfill  \begin{minipage}[t]{100mm}  $V - E + F = 2$ \hfill Леонард Эйлер, 1751 \end{minipage}
 \item Нормальное распределение \hfill \begin{minipage}[t]{100mm} $\Phi(x) = \frac{1}{\sqrt{2\pi\rho}}e^{\frac{(x-\rho)^2}{2\rho^2}}$ \hfill Карл Фридрих Гаусс, 1810 \end{minipage}
 \item Волновое уравнение \hfill \begin{minipage}[t]{100mm} $\frac{\partial^{2} u}{\partial t^{2}} = c^{2} \frac{\partial^{2} u}{\partial x^{2}}$ \hfill Жан Лерон Д'Аламбер, 1746 \end{minipage}
 \item Преобразование Фурье \hfill \begin{minipage}[t]{110mm} $f(\omega) = \int\limits_\infty ^\infty f(x)e^{-2\pi ix \omega}dx$  \hfill Жан Батист Жозеф Фурье, 1822 \end{minipage}
 \item 
  \begin{minipage}[t]{35mm} Уравнение \\Навье-Стокса\\ \end{minipage}
  \hfill\begin{minipage}[t]{130mm}$\rho(\frac{\partial {\bf v}}{\partial t}+{\bf v * \bigtriangledown v})= -\bigtriangledown p +\bigtriangledown * {\bf T}+{\bf f} $ \hfill А. Навье и Дж. Стокс, 1845 \end{minipage}
   \item Уравнения Максвелла \hfill \begin{minipage}[t]{110mm}  \begin{minipage}[t]{30mm}  $\nabla\cdot{\bf E} =   \frac{\rho}{\varepsilon_0}$ \\ $\nabla\times{\bf E} = -\frac{1}{c} \frac{\partial{\bf H}}{\partial t}$ \\ \end{minipage}
  \begin{minipage}[t]{30mm} $\nabla\cdot{\bf H} = 0$ \\$\nabla\times{\bf H} = \frac{1}{c} \frac {E}{\partial t}$ \\
   \end{minipage}  \hfill Дж. Максвелл,  1865\end{minipage}
   \item \begin{minipage}[t]{45mm} Второй закон \\термодинамики\\ \end{minipage}
  \hfill
 \begin{minipage}[t]{100mm}  $dS \ge 0$ \hfill Людвиг Больцман, 1874 \end{minipage}
  \item Относительность и её теория \hfill \begin{minipage}[t]{100mm} $E = mc^{2}$ \hfill Альберт Эйнштейн, 1905
\end{minipage}
\item\begin{minipage}[t]{45mm} Уравнение Шрёдингера \end{minipage}  \hfill \begin{minipage}[t]{100mm} $i\hbar\frac{\partial}{\partial t} \Psi=H\Psi $
    \hfill Эрвин Шрёдингер, 1927 \end{minipage}
 \item Теория информации \hfill
  \begin{minipage}[t]{100mm} $H = - \sum p(x)\log p(x)$ \hfill Клод Шеннон, 1949
   \end{minipage}
 \item Теория хаоса  \hfill \begin{minipage}[t]{100mm} $x_{t+1} = kx_{t}(1-x_{t})$  \hfill Роберт Мэй, 1975 \end{minipage}
   \item \begin{minipage}[t]{45mm} Уравнение \\Блэка — Шоулза\\ \end{minipage}
  \hfill 
 \begin{minipage}[t]{110mm}  $\frac{1}{2} S^{2}\sigma \frac{\partial^{2} V}{\partial S^{2}} +rS\frac{\partial V}{\partial S} +\frac{\partial V}{\partial t} -rV=0 $ \hfill Блэк,Шоулз, 1990 \end{minipage}
\end{enumerate}
\end{document}
